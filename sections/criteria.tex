\section{Karkasų vertinimas}\label{sec:criteria}

\Glsplwhom{framework} vertinimui apibrėšime kriterijus, kurie padės objektyviai išrinkti
realioms \glswhom{adversarial} tinkančius \glsplwhat{framework}. Kriterijus
laikysime esant dviejų tipų: \textbf{kokybiniai} (\glsshort{framework} atitinka
kriterijų arba ne) ir \textbf{kiekybiniai} (\glsshort{framework} atitinka
kriterijų dalinai: 0 \% -- visiškai neatitinka, 100 \% -- visiškai atitinka). \\

\textbf{Kriterijai} (surikiuoti pagal svarbą mažėjančia tvarka):
\vspace{-5pt}
\begin{enumerate}[label=K-\arabic*., ref=K-\arabic*]
    \item \Glsshort{framework} pritaikytas \enquote{juodos dėžės} atvejams (\textbf{kokybinis}).\label{enum:criteria:blackbox}
    \item \Glsshort{framework} pritaikytas komerciniams detektoriams (\textbf{kokybinis}).\label{enum:criteria:commercial}
    \item \Glsshort{framework} užtikrina realistišką atakos efektyvumo įvertinimą -- naudoja \glswhat{surrogateModel} (\textbf{kokybinis}).\label{enum:criteria:surrogate}
    \item \Glsshort{framework} užtikrina atakos efektyvumą (\textbf{kiekybinis}).\label{enum:criteria:effective}
\end{enumerate}

\newenvironment{criteriaTable}{
    \newcommand{\rowLast}[1]{##1}
    \newcommand{\row}[1]{##1 \\}
    \newcommand{\tbl}[1]{\gdef\Table{##1}}

    \def\Table{}
}{
    \begin{table}[h]
        \centering
        \begin{tabular}{|l|c|c|c|S|}
            \row{
            \textbf{\Glsshort{framework}}           &
            \textbf{\ref{enum:criteria:blackbox}}   &
            \textbf{\ref{enum:criteria:commercial}} &
            \textbf{\ref{enum:criteria:surrogate}}  &
                \textbf{\ref{enum:criteria:effective}}
            } \midrule
            \Table{}
        \end{tabular}
        \caption{\Glsplwhom{framework} vertinimas pagal kriterijus. \Glspl{framework} surikiuoti pagal įvertinimą mažėjančia tvarka.}\label{tab:criteria}
    \end{table}
}

\begin{criteriaTable}
    % \begin{tabular}{cols}
    \tbl{
        \row{ \refFramework{MalFox}        & \cmark{} & \cmark{} & \cmark{} & 56,00 \%}
        \row{ \refFramework{MalInfo}       & \cmark{} & \cmark{} & \xmark{} & 59,40 \%}
        \row{ \refFramework{AIMED}         & \cmark{} & \cmark{} & \xmark{} & 47,98 \%}
        \row{ \refFramework{N-gram MalGAN} & \cmark{} & \xmark{} & \cmark{} & 88,58 \%}
        \row{ \refFramework{MalGAN}        & \cmark{} & \xmark{} & \cmark{} & 81,00 \%}
        \row{ \refFramework{GAMMA}         & \cmark{} & \xmark{} & \xmark{} & 53,00 \%}
        \rowLast{ \refFramework{DQEAF}     & \cmark{} & \xmark{} & \xmark{} & 46,56 \%}
    }
    % \end{tabular}
\end{criteriaTable}

Kriterijų vertinimo lentelėje pastebime, jog dviejų aukščiausiai pagal \ref{enum:criteria:effective} kriterijų įvertintų \glsplwhom{framework} (\refFramework{N-gram MalGAN} ir \refFramework{MalGAN}) įvertinimai žymiai aukštesni, nei vidurkis (nuokrypis per 26,79 \% ir 19,21 \% atitinkamai), tačiau bendras jų įvertinimas pakankamai žemas (4 ir 5 vietos). Tai lemia \ref{enum:criteria:commercial} kriterijaus įvertinimas. Kadangi vertinimo tikslas yra nustatyti realioms \glswhom{adversarial} tinkančius \glsplwhat{framework}, negalime teigti, jog \glspl{framework}, puikiai konstruojantys \acs{ae} akademiniams modeliams, gebės konstruoti tokios pačios kokybės \acs{ae} komerciniams modeliams. Tai ir parodysime \ref{sec:experiment}-ame skyriuje. 