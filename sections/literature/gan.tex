\subsection{GAN tipo modelių \glspl{framework}}\label{sec:literature:gan}

\acs{gan} modeliai paremti Generatyviniais Priešiškais Tinklais (\aclp{gan}), kurių veikimo principas yra du neuroniniai tinklai (generatorius ir diskriminatorius), žaidžiantys \gls{zeroSumGame} \citeplace. Kenkėjiško kodo obfuskacijos kontekste ir ypač \enquote{juodos dėžės} atvejais, diskriminatorius atlieka \gls{surrogateModel} vaidmenį. Bendras \ac{gan} modelių mokymosi etapas yra tokia seka:
\begin{enumerate}
    \item Generatorius, naudodamas požymių vektorių ir tokios pačios dimensijos
          \enquote{triukšmo} (\textit{angl. noise}) vektorių, sugeneruoja perturbacijas.
    \item Originali kenkėjiška programa modifikuojama pagal perturbacijas (sukuriamas
          \ac{ae}).
    \item Diskriminatorius bando klasifikuoti sugeneruotą \ac{ae} (kenkėjiškas /
          nekenkėjiškas). Diskriminatoriaus klasifikacija lyginama su tikro detektoriaus
          klasifikacija. Jei ji teisinga -- atnaujinami generatoriaus parametrai pagal
          generatoriaus nuostolių funkciją. Kitu atveju, atnaujinami diskriminatoriaus
          parametrai pagal diskriminatoriaus nuostolių funkciją.
    \item Visa seka kartojama nustatytą kiekį kartų.
\end{enumerate} \citeplace.

\begin{describeModel}{MalGAN}{\cite{huGeneratingAdversarialMalware2017}}
    \purpose{
        Efektyviai išvengti \acs{ae} aptikimo, kai \acs{ml} kenkėjiškų programų detektoriaus implementacija nežinoma (\enquote{juodos dėžės} atvejis)
    }
    \surrogate{
        Daugiasluoksnis tiesioginio sklidimo neuroninis tinklas -- klasifikatorius. Įvestis -- programos požymių vektorius. Išvestis -- klasifikacija į kenksmingą arba nekensmingą. Šis tinklas taip pat naudojamas kaip diskriminatorius \acs{gan} architektūroje.
    }
    \mainModel{
        Daugiasluoksnis tiesioginio sklidimo neuroninis tinklas. Įvestis -- programos požymių vektorius ir tokios pačios dimensijos \enquote{triukšmo} vektorius. Išvestis -- modifikuotas požymių vektorius. Šis tinklas naudojamas kaip generatorius \acs{gan} architektūroje.
    }
    \features{
        \item \enquote{MalGan} straipsnyje naudojami tik \acs{api} vardų požymiai, patenkantys į PE formato programų požymių kategoriją (žr. \ref{sec:literature:features:pe}), tačiau autoriai nurodo, jog gali būti naudojami bet kokie požymiai.
    }
    \perturbations{
        \item Semantinės perturbacijos (\ref{sec:literature:perturbations:semantic}) -- nereikalingų \acs{api} vardų požymių pridėjimas
    }
\end{describeModel}