\subsection{GAN tipo modelių \glspl{framework}}\label{sec:literature:gan}

\acs{gan} modelių karkasai paremti Generatyviniais Priešiškais Tinklais (\aclp{gan}), kurių veikimo principas yra du neuroniniai tinklai (generatorius ir diskriminatorius), žaidžiantys \gls{zeroSumGame} \citeplace. Kenkėjiško kodo obfuskacijos kontekste ir ypač \enquote{juodos dėžės} atvejais, diskriminatorius atlieka \gls{surrogateModel} vaidmenį. Bendras \ac{gan} modelių mokymosi etapas yra tokia seka:
\begin{enumerate}
    \item Generatorius, naudodamas požymių vektorių ir tokios pačios dimensijos
          \enquote{triukšmo} (\angl{noise}) vektorių, sugeneruoja perturbacijas.
    \item Originali kenkėjiška programa modifikuojama pagal perturbacijas (sukuriamas
          \ac{ae}).
    \item Diskriminatorius bando klasifikuoti sugeneruotą \ac{ae} (kenkėjiškas /
          nekenkėjiškas). Diskriminatoriaus klasifikacija lyginama su tikro detektoriaus
          klasifikacija. Jei ji teisinga -- atnaujinami generatoriaus parametrai pagal
          generatoriaus nuostolių funkciją. Kitu atveju, atnaujinami diskriminatoriaus
          parametrai pagal diskriminatoriaus nuostolių funkciją.
    \item Visa seka kartojama nustatytą kiekį kartų.
\end{enumerate} \citeplace.

\begin{describeFramework}{MalGAN}{\cite{huGeneratingAdversarialMalware2017}}
    \purpose{
        Efektyviai išvengti \acs{ae} aptikimo, kai \acs{ml} kenkėjiškų programų detektoriaus implementacija nežinoma (\enquote{juodos dėžės} atvejis)
    }
    \surrogate{
        Daugiasluoksnis tiesioginio sklidimo neuroninis tinklas -- klasifikatorius. Įvestis -- programos požymių vektorius. Išvestis -- klasifikacija į kenksmingą arba nekenksmingą. Šis tinklas taip pat naudojamas kaip diskriminatorius \acs{gan} architektūroje.
    }
    \mainModel{
        Daugiasluoksnis tiesioginio sklidimo neuroninis tinklas. Įvestis -- programos požymių vektorius ir tokios pačios dimensijos \enquote{triukšmo} vektorius. Išvestis -- modifikuotas požymių vektorius. Šis tinklas naudojamas kaip generatorius \acs{gan} architektūroje.
    }
    \features{}{
        \item \textit{MalGan} straipsnyje naudojami tik \acs{api} vardų požymiai, patenkantys į PE formato programų požymių kategoriją (žr. \ref{sec:literature:features:pe}), tačiau autoriai nurodo, jog gali būti naudojami bet kokie požymiai\footnote{\label{footnote:detector-assumptions}autoriai nagrinėja \enquote{juodos dėžės} atvejį su prielaida, jog detektoriaus naudojami požymiai yra žinomi.}.
    }
    \perturbations{}{
        \item Semantinės perturbacijos (\ref{sec:literature:perturbations:semantic}) --
        nereikalingų \acs{api} vardų požymių pridėjimas }
\end{describeFramework}

\begin{describeFramework}{N-gram MalGAN}{\cite{zhuNgramMalGANEvading2022}}
    \purpose{
        Supaprastinti, pagreitinti ir pagerinti \glswhat{adversarial}. Pašalinti prielaidas\footnoteref{footnote:detector-assumptions} apie detektorių \enquote{juodos dėžės} atvejais.
    }
    \surrogate{
        Surogatinio modelio veikimas ir architektūra tokia pati, kaip ir \refFramework{MalGAN}
    }
    \mainModel{
        Pagrindinio modelio veikimas ir architektūra labai panašūs į \refFramework{MalGAN}, tačiau norėdami stabilizuoti mokymosi procesą, autoriai siūlo nenaudoti \enquote{triukšmo} vektoriaus. Vietoje to, generatoriaus išvestis ($n$-matis vektorius) modifikuojama nekeičiant pirmų $m$ dimensijų, o kitas $n-m$ pakeičiant nekenksmingų programų požymiais.
    }
    \features{}{
        \item Baitų lygio požymiai (\ref{sec:literature:features:byte}) -- $n$-gramos. }
    \perturbations{Autoriai neatliko eksperimentų su perturbuotomis programomis,
        tačiau pažymi, jog norint gauti sugeneruotus požymių vektorius užtenka pridėti
        reikiamus baitus programos gale. Tai atitinka
        \ref{sec:literature:perturbations:byte} apibrėžtą \vspace{5pt}}{
        \item baitų lygio perturbaciją \textit{ARBE}, tik šiuo atveju pridedami baitai nebūtų
        atsitiktiniai, o norimos $n$-gramos. }
\end{describeFramework}

\begin{describeFramework}{MalFox}{\cite{zhongMalFoxCamouflagedAdversarial2024}}
    \purpose{Generuoti \acs{ae}, kurių neaptiktų komerciniai detektoriai (prieš tai aptarti \glspl{framework} eksperimentams kaip nepriklausomą detektorių naudojo tokius \acs{ml} modelius, kaip \acs{svm}, \acs{knn}, \acs{gbdt} ir kt., bet ne komercinius detektorius). Šio \glswhom{framework} detektorius yra \textit{VirusTotal} (viešai prieinama paslauga, agreguojanti virš 70 komercinių kenkėjiškų programų detektorių).}
    \surrogate{Surogatinis modelis, kaip ir kituose \acs{gan} tipo modelių karkasuose, naudojamas kaip diskriminatorius. Įvestis -- perturbuota programa. Išvestis -- klasifikacija į kenksmingą arba nekenksmingą. Implementacija -- konvoliucinis neuroninis tinklas (\acs{cnn}).}
    \mainModel{Standartinis \acs{gan} generatorius, požymių vektorių sujungiantis su \enquote{triukšmo} vektoriumi. Implementacija -- konvoliucinis neuroninis tinklas (\acs{cnn}).}
    \features{}{
        \item PE formato programų požymiai (\ref{sec:literature:features:pe}) -- \acs{dll}
        vardai. } \perturbations{}{
        \item Visos kompleksinės perturbacijos (\ref{sec:literature:perturbations:complex}) }
\end{describeFramework}