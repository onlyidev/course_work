\subsection{GAN tipo modelių karkasai}\label{sec:literature:gan}

\ac{gan} modeliai paremti Generatyviniais Priešiškais Tinklais (\aclp{gan}), kurių veikimo principas yra du neuroniniai tinklai (generatorius ir diskriminatorius), žaidžiantys \gls{zeroSumGame} \citeplace. Kenkėjiško kodo obfuskacijos kontekste ir ypač \enquote{juodos dėžės} atvejais, diskriminatorius atlieka \gls{surrogateModel} vaidmenį. Bendras \ac{gan} modelių mokymosi etapas yra tokia seka:
\begin{enumerate}
    \item generatorius, naudodamas požymių vektorių ir tokios pačios dimensijos \enquote{triukšmo} (\textit{angl. noise}) vektorių, sugeneruoja perturbacijas
    \item originali kenkėjiška programa modifikuojama pagal perturbacijas (sukuriamas \ac{ae})
    \item diskriminatorius bando klasifikuoti sugeneruotą \ac{ae} (kenkėjiškas / nekenkėjiškas). Diskriminatoriaus klasifikacija lyginama su tikro detektoriaus klasifikacija. Jei ji teisinga -- atnaujinami generatoriaus parametrai pagal generatoriaus nuostolių funkciją. Kitu atveju, atnaujinami diskriminatoriaus parametrai pagal diskriminatoriaus nuostolių funkciją.
    \item visa seka kartojama nustatytą kiekį kartų
\end{enumerate} \citeplace.