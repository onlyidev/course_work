\subsection{Skatinamojo mokymosi tipo modelių \glspl{framework}}\label{sec:literature:rl}

Skatinamojo mokymosi (\acs{rl}) modeliai susideda iš agento ir aplinkos.
Aplinka susideda iš informatyvių požymių ištraukimo metodo (\textit{angl.
    feature extraction}) ir kenkėjiškų programų detektoriaus. Šiuo atveju aplinkos
būsenų erdvė $S$ yra požymių vektorių erdvė. Agentas -- tai algoritmas ar
neuroninis tinklas, kurio tikslas yra surasti optimalią strategiją
\gls{policy}. Šiuo atveju strategijos veiksmų erdvė $A$ susideda iš
perturbacijų (žr. \ref{sec:literature:perturbations}) \cite{fangEvadingMalwareEngines2019}. Bendras
\ac{rl} modelių mokymosi etapas yra tokia seka \cite{fangEvadingMalwareEngines2019, zhongReinforcementLearningBased2022}:
\begin{enumerate}
    \item Agentas, naudodamas dabartinę aplinkos būseną ir praeito veiksmo atlygį
          (\angl{reward}), parenka sekantį veiksmą iš galimų veiksmų aibės ir taiko
          mokymosi algoritmą (algoritmas priklauso nuo agento įgyvendinimo).
    \item Atliekamas veiksmas -- perturbuojama programa arba požymių vektorius (priklauso
          nuo \glswhom{framework}).
    \item Gaunami aplinkos kitimo įverčiai -- nauja būsena ir atlygis, skaičiuojamas
          pagal detektoriaus klasifikacijos rezultatą.
    \item Seka kartojama tol, kol agentas nelaiko strategijos optimalia arba nustatytą
          kiekį kartų.
\end{enumerate}

\begin{describeFramework}{DQEAF}{\cite{fangEvadingMalwareEngines2019}}
    \introLastPar{
        Šis \glsshort{framework} taiko gilųjį skatinamąjį mokymąsi, kai agentas implementuojamas kaip gilusis neuroninis tinklas.
    }
    \purpose{Parodyti, jog \acs{ml} kenkėjiškų programų detektoriai, ypač modeliai, treniruoti prižiūrimu mokymusi, yra pažeidžiami \glswhom{adversarial}}
    \surrogate{\acs{rl} karkasuose nenaudojami surogatiniai modeliai. Kaip \enquote{juodos dėžės} detektorius pasirinktas \acs{gbdt} modelis.}
    \mainModel{Agentas implementuotas kaip gilusis $Q$-tinklas (\acs{cnn} praplėtimas, kai tinklas naudojamas kaip \glswhom{qfunction} aproksimacija). Taip pat taikomas prioritetizuotas patirčių pakartojimo metodas (\angl{prioritized experience replay}), kuomet agentas treniruojamas tik su aukštą atlygį gavusiais perėjimais ($S \times A$).}
    \features{Požymių vektorius taip pat apibrėžia visų būsenų erdvę $S$. Šiuo atveju $S = \mathbb{R}^{513}$.\vspace{5pt}}{
        \item Baitų lygio požymiai (\ref{sec:literature:features:byte}) -- baitų/entropijos
        histograma. } \perturbations{Perturbacijos apibrėžia visų galimų agento veiksmų
        erdvę $A$. Šiuo atveju $A = \set{0,1}^4$.\vspace{-10pt}}{
        \item Baitų lygio perturbacijos (\ref{sec:literature:perturbations:byte})
        \begin{itemize}
            \item \textit{ARBE}
            \item \textit{ARI}
            \item \textit{ARS}
            \item \textit{RS}
        \end{itemize}
    }
\end{describeFramework}

\begin{describeFramework}{MalInfo}{\cite{zhongReinforcementLearningBased2022}}
    \introLastPar{
        \textit{\Name} remiasi \refFramework{MalFox}.
    }
    \purpose{Surasti optimalią obfuskacijos strategiją konkrečiai programai, pagal kurią sukurtas \acs{ae} nebūtų aptiktas komercinių kenkėjiškų programų detektorių.}
    \surrogate{\acs{rl} surogatinis modelis nenaudojamas. \enquote{Juodos dėžės} detektoriumi pasirinkti komerciniai detektoriai (\enquote{VirusTotal})}
    \mainModel{Agentas implementuotas kaip klasikiniai \acs{ml} algoritmai (konkrečiai dinaminis programavimas ir skirtumų laike (\angl{temporal difference}) algoritmas).}
    \features{Agentas nėra neuroninis tinklas ir požymių iš programos netraukia. Agentas mokosi tik iš perėjimų, o būsenų erdvė $S$ yra originali programa ir perturbuoti jos variantai. Teoriškai perturbuotų programos variantų galėtų būti be galo daug, tuomet $S = A^\infty, |S| = \aleph_0$, tačiau autoriai nurodo, jog daugiau nei 3 sluoksniai kompleksinių perturbacijų reikšmingai paveikia programos veikimo laiką, o tai gali \enquote{sukelti įtarimų} komerciniams detektoriams. Todėl pasirinkta $S = A^3$ \vspace{-20pt}}{\item[]}
    \perturbations{$A = \set{0,1,2,3}$ \vspace{5pt}}{
        \item \enquote{Null} perturbacija -- naudinga tik formaliam pilnumui (atitinka nulinį $A$ veiksmą)
        \item Visos kompleksinės perturbacijos (\ref{sec:literature:perturbations:complex}),
        t.~y. tokios pačios, kaip ir \refFramework{MalFox} \glswhom{framework}. }
\end{describeFramework}