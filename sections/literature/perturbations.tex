\subsection{Perturbacijos}\label{sec:literature:perturbations}

Perturbacijos -- tai pagrindinis obfuskacijos metodas \ac{ae} kūrimui. Perturbacijų tikslas yra pakeisti kenkėjiškos programos veikimą išsaugant originalų funkcionalumą. Perturbacijos gali būti sudėtingos ir apimti visą programą (pvz. visos programos užšifravimas ir pridėjimas prie kitos programos), semantinės (pvz. tam tikrų mašininio kodo instrukcijų keitimas į ekvivalentų rezultatą pasiekiančias) arba baitų lygio (pvz. nulinių baitų pridėjimas programos gale) \citeplace. Perturbacijų parinkimas įeina į karkaso apibrėžimą. Šioje sekcijoje aptariamos mokslinėje literatūroje minimos perturbacijos.
\begin{itemize}
    \item 
\end{itemize}