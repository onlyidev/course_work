\subsection{Naudojami požymiai}\label{sec:literature:features}
\color{red}
Pirmasis veiksmas treniruojant ar naudojant \ac{ml} modelį, paremtą neuroniniais tinklais, yra paversti įvesties duomenis į požymių vektorių \citeplace. Kenkėjiško kodo obfuskacijos kontekste, požymių pasirinkimas nėra vienareikšmis, o įeina į karkaso apibrėžimą. Šioje sekcijoje apžvelgiami ir klasifikuojami \acs{gan}, \acs{rl} ir \acs{genetic} modelių tipų karkasų naudojami požymiai.
\color{black}
\begin{itemize}
    \item \textbf{\acs{dll} vardai (arba \acs{api} vardai \cite{huGeneratingAdversarialMalware2017})} \cite{zhongMalFoxCamouflagedAdversarial2024}. \acs{pe} faile turi būti nurodyti visi naudojami \acs{dll}. Prieš pradedant treniruoti \acs{ml} modelį, atliekama visų turimų programų analizė ir nustatoma visų naudojamų \acs{dll} aibė $D$. Tarkime $|D| = n$. Tuomet, požymių vektorius programai, naudojančiai $X \subseteq D$ \acs{dll}, bus $n$-matis dvejetainis vektorius, kurio $i$-asis elementas yra $\begin{cases}
        0, \text{ jei } D_i \not \in X,\\
        1, \text{ jei } D_i \in X.
    \end{cases}$ Čia $D_i$ -- $i$-asis $D$ elementas.
    \item \textbf{$n$-gramos} \cite{zhuNgramMalGANEvading2022}. Dažniausiai sutinkamos skaitmeniniame natūraliosios kalbos apdorojime (\acs{nlp}). Tai yra $n$ žodžių junginiai, arba, sukompiliuotų programų apdorojimo kontekste, $n$ baitų junginiai. Nustatant požymių vektorių, visos $n$-gramos surikiuojamos pagal pasikartojimą programoje mažėjimo tvarka (\enquote{populiariausios} viršuje). Iš pirmų $m$ reikšmių sudaromas $m$-matis vektorius -- tai ir yra požymių vektorius.
    \item \textbf{Baitų/entropijos histograma} \cite{saxeDeepNeuralNetwork2015}. Specifinis metodas, užkoduojantis dažniausiai pasikartojančias baitų ir entropijos poras $256$ dimensijų vektoriumi.
    \item \textbf{\acs{pe} metaduomenys} \cite{andersonLearningEvadeStatic2018}. Tai visi \acs{pe} formate faile esantys metaduomenys, tokie, kaip sekcijų pavadinimai, sekcijų dydžiai, \enquote{ImportTable} ir \enquote{ExportTable} metaduomenys ir kt. Formuojant požymių vektorių skaičiuojama metaduomenų \gls{hashfunction}.
    \item \textbf{Prasmingų žodžių (angl. Strings) kiekis} \cite{andersonLearningEvadeStatic2018}. Prasmingus žodžius suprantame kaip turinčius prasmę žmogui \textit{(angl. Human Readable)}. Tai gali būti URL, failų keliai \textit{(angl. File Paths)} ar registro raktų pavadinimai. Kadangi prasmingų žodžių kiekis tėra vienas skaičius, požymių dažniausiai formuojamas prijungiant ir kitus požymius.
\end{itemize}