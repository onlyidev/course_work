\section{Literatūros apžvalga}\label{sec:literature}

Mokslinėje literatūroje kenkėjiško kodo ir programų obfuskacijai daugiausia dėmesio skiriama \ac{ae} generavimui, dėl paplitusio \acs{ml} modelių naudojimo kenkėjiško kodo ir programų detekcijos srityje \citeplace. Dažniausia atakų platforma -- operacinė sistema \enquote{Windows} ir jos naudojami \acs{pe} formato failai \citeplace.

Kenkėjiškų programų analizė gali būti išskiriama į \textbf{statinę analizę} ir \textbf{elgesio (\angl{behavior}) analizę}. Statinės analizės detektoriai yra labiau paplitę ir prieinami, kadangi požymių išskyrimas iš tam tikro iš anksto žinomo formato failų ir jų klasifikavimas yra sąlyginai nesudėtinga užduotis. Tuo tarpu elgesio analizės detektoriai dažniausiai neprieinami eiliniams vartotojams, o diegiami korporacijose \citeplace. Jų veikimas pagrįstas nuolatiniu sistemos stebėjimu ir anomalijų detekcija, tad yra kartu ir sudėtingesnė, ir reikalaujanti daugiau resursų, nei statinė analizė, užduotis.

Mokslinėje literatūroje nagrinėjant specifinį būdą generuoti \acs{ae}, apibrėžiamas \gls{framework}. \Glswhat{framework} sudaro
\begin{itemize}
    \item \Glswhom{surrogateModel} tipas ir architektūra
    \item \acs{ml} modelio tipas ir architektūra
    \item \acs{ml} modelio naudojami požymiai (žr. \ref{sec:literature:features})
    \item \acs{ml} modelio naudojamos perturbacijos (žr. \ref{sec:literature:perturbations})
\end{itemize}

\subsection{Naudojami požymiai}\label{sec:literature:features}
\color{red}
Pirmasis veiksmas treniruojant ar naudojant \ac{ml} modelį, paremtą neuroniniais tinklais, yra paversti įvesties duomenis į požymių vektorių \citeplace. Kenkėjiško kodo obfuskacijos kontekste, požymių pasirinkimas nėra vienareikšmis, o įeina į karkaso apibrėžimą. Šioje sekcijoje apžvelgiami ir klasifikuojami \acs{gan}, \acs{rl} ir \acs{genetic} modelių tipų karkasų naudojami požymiai.
\color{black}
\begin{itemize}
    \item \textbf{\acs{dll} vardai (arba \acs{api} vardai \cite{huGeneratingAdversarialMalware2017})} \cite{zhongMalFoxCamouflagedAdversarial2024}. \acs{pe} faile turi būti nurodyti visi naudojami \acs{dll}. Prieš pradedant treniruoti \acs{ml} modelį, atliekama visų turimų programų analizė ir nustatoma visų naudojamų \acs{dll} aibė $D$. Tarkime $|D| = n$. Tuomet, požymių vektorius programai, naudojančiai $X \subseteq D$ \acs{dll}, bus $n$-matis dvejetainis vektorius, kurio $i$-asis elementas yra $\begin{cases}
        0, \text{ jei } D_i \not \in X,\\
        1, \text{ jei } D_i \in X.
    \end{cases}$ Čia $D_i$ -- $i$-asis $D$ elementas.
    \item \textbf{$n$-gramos} \cite{zhuNgramMalGANEvading2022}. Dažniausiai sutinkamos skaitmeniniame natūraliosios kalbos apdorojime (\acs{nlp}). Tai yra $n$ žodžių junginiai, arba, sukompiliuotų programų apdorojimo kontekste, $n$ baitų junginiai. Nustatant požymių vektorių, visos $n$-gramos surikiuojamos pagal pasikartojimą programoje mažėjimo tvarka (\enquote{populiariausios} viršuje). Iš pirmų $m$ reikšmių sudaromas $m$-matis vektorius -- tai ir yra požymių vektorius.
    \item \textbf{Baitų/entropijos histograma} \cite{saxeDeepNeuralNetwork2015}. Specifinis metodas, užkoduojantis dažniausiai pasikartojančias baitų ir entropijos poras $256$ dimensijų vektoriumi.
    \item \textbf{\acs{pe} metaduomenys} \cite{andersonLearningEvadeStatic2018}. Tai visi \acs{pe} formate faile esantys metaduomenys, tokie, kaip sekcijų pavadinimai, sekcijų dydžiai, \enquote{ImportTable} ir \enquote{ExportTable} metaduomenys ir kt. Formuojant požymių vektorių skaičiuojama metaduomenų \gls{hashfunction}.
    \item \textbf{Prasmingų žodžių (angl. Strings) kiekis} \cite{andersonLearningEvadeStatic2018}. Prasmingus žodžius suprantame kaip turinčius prasmę žmogui \textit{(angl. Human Readable)}. Tai gali būti URL, failų keliai \textit{(angl. File Paths)} ar registro raktų pavadinimai. Kadangi prasmingų žodžių kiekis tėra vienas skaičius, požymių dažniausiai formuojamas prijungiant ir kitus požymius.
\end{itemize}
\subsection{Perturbacijos}\label{sec:literature:perturbations}

Perturbacijos -- tai pagrindinis obfuskacijos metodas \ac{ae} kūrimui.
Perturbacijų tikslas yra pakeisti kenkėjiškos programos veikimą išsaugant
originalų funkcionalumą. Perturbacijos gali būti sudėtingos ir apimti visą
programą (pvz., visos programos užšifravimas ir pridėjimas prie kitos
programos), semantinės (pvz., tam tikrų mašininio kodo instrukcijų keitimas į
ekvivalentų rezultatą pasiekiančias) arba baitų lygio (pvz., nulinių baitų
pridėjimas programos gale) \cite{huGeneratingAdversarialMalware2017}. Perturbacijų parinkimas įeina į
\glswhom{framework} apibrėžimą. Šiame poskyryje aptariamos mokslinėje
literatūroje minimos perturbacijos.
\subsubsection{Baitų lygio perturbacijos}\label{sec:literature:perturbations:byte}
Pačias paprasčiausias baitų lygio perturbacijas galima taikyti bet kokio formato failams, tačiau labiau prasmingos perturbacijos taikomos \acs{pe} formato failams. Išskiriamos šios pagrindinės baitų lygio perturbacijos:
\begin{itemize}
    \item \textbf{\textit{ARBE} (\textit{Append Random Bytes at the End})} \cite{fangEvadingMalwareEngines2019}. \acs{pe} formato failo gale pridedami atsitiktiniai baitai.
    \item \textbf{\textit{ARI} (\textit{Append Random Import})} \cite{fangEvadingMalwareEngines2019}. \acs{pe} formato failo \textit{ImportAddressTable} lentelėje pridedama atsitiktinai pavadinta biblioteka su atsitiktinai pavadinta funkcija.
    \item \textbf{\textit{ARS} (\textit{Append Randomly named Section})} \cite{fangEvadingMalwareEngines2019}. \acs{pe} formato failo \textit{SectionTable} lentelėje pridedamos atsitiktinės sekcijos (sekcijos ir jų tipai yra apibrėžti \acs{pe} formate).
    \item \textbf{\textit{RS} (\textit{Remove Signature})} \cite{fangEvadingMalwareEngines2019}. Sertifikato pašalinimas iš \acs{pe} formato failo \textit{CertificateTable} lentelės.
    \item \textbf{Naujas įeities taškas} \cite{andersonLearningEvadeStatic2018}. Prasidėjus programai, iškart peršokama nuo naujo įeities taško į originalųjį.
    \item \textbf{\textit{Header Fields}} \cite{demetrioAdversarialEXEmplesSurvey2021}. \acs{pe} formato failo \textit{PE Header} ir \textit{Optional Header} dalių specifinių laukų keitimas (pvz., sekcijos pavadinimo keitimas \cite{andersonLearningEvadeStatic2018}).
    \item \textbf{\textit{Partial DOS}} \cite{demetrioAdversarialEXEmplesSurvey2021}. \acs{pe} formato failo \textit{DOS Header} dalies pirmi 58 baitai po \textit{MZ} skaičiaus yra nenaudojami moderniose operacinėse sistemose, tad juos galima keisti.
    \item \textbf{\textit{Slack Space}} \cite{demetrioAdversarialEXEmplesSurvey2021}. Dėl \acs{pe} formato specifikos, kiekviena nauja sekcija turi prasidėti tam tikro skaičiaus, nurodyto \textit{PE Header} dalyje, kartotiniu nuo pradžios. Kompiliatoriai šį reikalavimą išpildo sekcijų gale pridėdami tiek nulinių baitų, kiek reikia teisingam sulygiavimui pasiekti. Būtent ši nulinių baitų erdvė gali būti keičiama be jokios įtakos originaliai programai.
    \item \textbf{\textit{Padding}} \cite{demetrioAdversarialEXEmplesSurvey2021}. Nulinių baitų pridėjimas failo gale.
    \item \textbf{\textit{Full DOS}} \cite{demetrioAdversarialEXEmplesSurvey2021}. Perturbacijos esmė tokia pat, kaip ir \textit{Partial DOS}, tik naudojami visi \textit{DOS} dalies baitai, išskyrus \textit{MZ} ir \textit{PE Offset} (\textit{Partial DOS} manipuliacijoms naudoja tik dalį tarp \textit{MZ} ir \textit{PE Offset}).
    \item \textbf{\textit{Extend}} \cite{demetrioAdversarialEXEmplesSurvey2021}. Pakeičiama \acs{pe} formato faile \textit{DOS} dalyje esanti \textit{PE Offset} reikšmė į didesnę\footnote{\label{footnote:structure}šios reikšmės padidinimas reiškia visos failo struktūros keitimą (\textit{DOS} dalis yra failo pradžioje). Būtina pakeisti visų sekcijų vietas nuo pradžios (\angl{offset}) jų metaduomenyse.}. Taip padidinama (išplečiama) visa \textit{DOS} dalis. Tolesnis perturbacijos principas yra toks pat, kaip ir \textit{Full DOS}.
    \item \textbf{\textit{Shift}} \cite{demetrioAdversarialEXEmplesSurvey2021}. \acs{pe} formato failuose kiekvienas sekcijos blokas prasideda su sekcijos vieta nuo pradžios (\angl{offset}). Tarkime ši reikšmė yra $S$. Sekcijos kodas pradedamas vykdyti tik nuo adreso $P+S$, kur $P$ -- programos pradžios adresas. Vadinasi, padidinus\footnoteref{footnote:structure} $S$ per $n$, atsiranda $n$ baitų laisvos vietos iki sekcijos pradžios, kurią galima keisti be jokios įtakos programos veikimui.
\end{itemize}
\subsubsection{Semantinės perturbacijos}\label{sec:literature:perturbations:semantic}
Semantinių perturbacijų įgyvendinimas taip pat atliekamas baitų lygyje, tačiau šie pokyčiai turi aukštesnio lygio prasmę. Išskiriamos šios semantinės perturbacijos:
\begin{itemize}
    \item \textbf{Nereikalingų \hyperref[feature:dll]{DLL/API vardų} požymių pridėjimas} \cite{huGeneratingAdversarialMalware2017}. \acs{pe} formato faile \textit{ImportTable} lentelėje pridedami originalios programos nenaudojami \acs{dll}/\acs{api} vardai.
    \item \textbf{\textit{Binary Rewriting}} \cite{demetrioAdversarialEXEmplesSurvey2021}. Semantinis instrukcijų perrašymas. Pavyzdžiui, $A+B$ instrukcijos pakeitimas į $A-(-B)$.
\end{itemize}

\subsubsection{Kompleksinės perturbacijos}\label{sec:literature:perturbations:complex}
Kompleksinės perturbacijos yra pritaikomos tam tikriems tikslams. Obfuskacijos ir \glswhose{adversarial} tikslams literatūroje minimos šios kompleksinės perturbacijos:
\begin{itemize}
    \item \textbf{\textit{Obfusmal}} \cite{zhongMalFoxCamouflagedAdversarial2024}. Užšifruojama originalios programos kodo sekcija. Sukuriama ir originalios programos gale pridedama programa \textit{Shell.dll}, kurioje laikomas atšifravimo raktas, originalios programos kodo sekcijos adresas ir dydis. Be to, \textit{Shell.dll} geba atšifruoti originalios programos kodo sekciją ir jai perduoti kontrolę. \textit{Shell.dll} pridedama prie naudojamų \acs{dll}, o programos pradžios taškas nustatomas į \textit{Shell.dll} pradžios tašką. Iliustracija pateikiama \ref{fig:perturbations}-ame pav.
    \item \textbf{\textit{Stealmal}} \cite{zhongMalFoxCamouflagedAdversarial2024}. Visa originali programa užšifruojama ir pridedama prie programos \textit{Shell.exe} galo. \textit{Shell.exe} geba atšifruoti originalią programą ir perduoti jai kontrolę. Iliustracija pateikiama \ref{fig:perturbations}-ame pav.
    \item \textbf{\textit{Hollowmal}} \cite{zhongMalFoxCamouflagedAdversarial2024}. Užšifruojama visa originali programa. Ji pridedama prie kurios nors nekenksmingos programos galo. Prie šio junginio galo pridedama \textit{Hollow.dll} programa, kurios veikiamas panašus į \textit{Shell.exe} iš \textit{Stealmal}. Viso junginio pradžios taškas nustatomas į \textit{Hollowmal.dll} pradžios tašką. Iliustracija pateikiama \ref{fig:perturbations}-ame pav.
\end{itemize}

\begin{figure}[h]
    \begin{small}
        \begin{center}
            \includegraphics[width=0.6\textwidth]{img/complex-perturbations.png}
        \end{center}
        \caption{Obfusmal (a), Stealmal (b) ir Hollowmal (c) perturbacijų veikimo principų iliustracijos. Adaptuota iš \cite{zhongReinforcementLearningBased2022}}\label{fig:perturbations}
    \end{small}
\end{figure}
\subsection{GAN tipo modelių \glspl{framework}}\label{sec:literature:gan}

\acs{gan} modeliai paremti Generatyviniais Priešiškais Tinklais (\aclp{gan}), kurių veikimo principas yra du neuroniniai tinklai (generatorius ir diskriminatorius), žaidžiantys \gls{zeroSumGame} \citeplace. Kenkėjiško kodo obfuskacijos kontekste ir ypač \enquote{juodos dėžės} atvejais, diskriminatorius atlieka \gls{surrogateModel} vaidmenį. Bendras \ac{gan} modelių mokymosi etapas yra tokia seka:
\begin{enumerate}
    \item Generatorius, naudodamas požymių vektorių ir tokios pačios dimensijos
          \enquote{triukšmo} (\textit{angl. noise}) vektorių, sugeneruoja perturbacijas.
    \item Originali kenkėjiška programa modifikuojama pagal perturbacijas (sukuriamas
          \ac{ae}).
    \item Diskriminatorius bando klasifikuoti sugeneruotą \ac{ae} (kenkėjiškas /
          nekenkėjiškas). Diskriminatoriaus klasifikacija lyginama su tikro detektoriaus
          klasifikacija. Jei ji teisinga -- atnaujinami generatoriaus parametrai pagal
          generatoriaus nuostolių funkciją. Kitu atveju, atnaujinami diskriminatoriaus
          parametrai pagal diskriminatoriaus nuostolių funkciją.
    \item Visa seka kartojama nustatytą kiekį kartų.
\end{enumerate} \citeplace.

\begin{describeFramework}{MalGAN}{\cite{huGeneratingAdversarialMalware2017}}
    \purpose{
        Efektyviai išvengti \acs{ae} aptikimo, kai \acs{ml} kenkėjiškų programų detektoriaus implementacija nežinoma (\enquote{juodos dėžės} atvejis)
    }
    \surrogate{
        Daugiasluoksnis tiesioginio sklidimo neuroninis tinklas -- klasifikatorius. Įvestis -- programos požymių vektorius. Išvestis -- klasifikacija į kenksmingą arba nekenksmingą. Šis tinklas taip pat naudojamas kaip diskriminatorius \acs{gan} architektūroje.
    }
    \mainModel{
        Daugiasluoksnis tiesioginio sklidimo neuroninis tinklas. Įvestis -- programos požymių vektorius ir tokios pačios dimensijos \enquote{triukšmo} vektorius. Išvestis -- modifikuotas požymių vektorius. Šis tinklas naudojamas kaip generatorius \acs{gan} architektūroje.
    }
    \features{}{
        \item \enquote{MalGan} straipsnyje naudojami tik \acs{api} vardų požymiai, patenkantys į PE formato programų požymių kategoriją (žr. \ref{sec:literature:features:pe}), tačiau autoriai nurodo, jog gali būti naudojami bet kokie požymiai\footnote{\label{footnote:detector-assumptions}Autoriai nagrinėja \enquote{juodos dėžės} atvejį su prielaida, jog detektoriaus naudojami požymiai yra žinomi.}.
    }
    \perturbations{}{
        \item Semantinės perturbacijos (\ref{sec:literature:perturbations:semantic}) --
        nereikalingų \acs{api} vardų požymių pridėjimas }
\end{describeFramework}

\begin{describeFramework}{N-gram MalGAN}{\cite{zhuNgramMalGANEvading2022}}
    \purpose{
        Supaprastinti, pagreitinti ir pagerinti \glswhat{adversarial}. Pašalinti prielaidas\footnoteref{footnote:detector-assumptions} apie detektorių \enquote{juodos dėžės} atvejais.
    }
    \surrogate{
        Surogatinio modelio veikimas ir architektūra tokia pati, kaip ir \refFramework{MalGAN}
    }
    \mainModel{
        Pagrindinio modelio veikimas ir architektūra labai panašūs į \refFramework{MalGAN}, tačiau norėdami stabilizuoti mokymosi procesą, autoriai siūlo nenaudoti \enquote{triukšmo} vektoriaus. Vietoje to, generatoriaus išvestis ($n$-matis vektorius) modifikuojama nekeičiant pirmų $m$ dimensijų, o kitas $n-m$ pakeičiant nekenksmingų programų požymiais.
    }
    \features{}{
        \item Baitų lygio požymiai (\ref{sec:literature:features:byte}) -- $n$-gramos. }
    \perturbations{Autoriai neatliko eksperimentų su perturbuotomis programomis,
        tačiau pažymi, jog norint gauti sugeneruotus požymių vektorius užtenka pridėti
        reikiamus baitus programos gale. Tai atitinka
        \ref{sec:literature:perturbations:byte} apibrėžtą \vspace{5pt}}{
        \item baitų lygio perturbaciją \enquote{ARBE}, tik šiuo atveju pridedami baitai
        nebūtų atsitiktiniai, o norimos $n$-gramos. }
\end{describeFramework}

\begin{describeFramework}{MalFox}{\cite{zhongMalFoxCamouflagedAdversarial2024}}
    \purpose{Generuoti \acs{ae}, kurių neaptiktų komerciniai detektoriai (prieš tai aptarti \glspl{framework} eksperimentams kaip nepriklausomą detektorių naudojo tokius \acs{ml} modelius, kaip \acs{svm}, \acs{knn}, \acs{gbdt} ir kt., bet ne komercinius detektorius)}
    \surrogate{Surogatinis modelis, kaip ir kituose \acs{gan} tipo modelių karkasuose, naudojamas kaip diskriminatorius. Įvestis -- perturbuota programa. Išvestis -- klasifikacija į kenksmingą arba nekenksmingą. Implementacija -- konvoliucinis neuroninis tinklas (\acs{cnn}).}
    \mainModel{Standartinis \acs{gan} generatorius, požymių vektorių sujungiantis su \enquote{triukšmo} vektoriumi. Implementacija -- konvoliucinis neuroninis tinklas (\acs{cnn}).}
    \features{}{
        \item PE formato programų požymiai (\ref{sec:literature:features:pe}) -- \acs{dll}
        vardai. } 
    \perturbations{}{ 
        \item Visos kompleksinės perturbacijos (\ref{sec:literature:perturbations:complex}) 
    }
\end{describeFramework}
\subsection{Skatinamojo mokymosi tipo modelių \glspl{framework}}\label{sec:literature:rl}

Skatinamojo mokymosi (\acs{rl}) modeliai susideda iš agento ir aplinkos.
Aplinka susideda iš informatyvių požymių ištraukimo metodo (\textit{angl.
    feature extraction}) ir kenkėjiškų programų detektoriaus. Šiuo atveju aplinkos
būsenų erdvė $S$ yra požymių vektorių erdvė. Agentas -- tai algoritmas ar neuroninis
tinklas, kurio tikslas yra surasti optimalią strategiją (\gls{policy}). Šiuo
atveju strategijos veiksmų erdvė $A$ susideda iš perturbacijų (žr.
\ref{sec:literature:perturbations}) \citeplace. Bendras \ac{rl} modelių
mokymosi etapas yra tokia seka:
\begin{enumerate}
    \item agentas, naudodamas dabartinę aplinkos būseną ir praeito veiksmo atlygį
          (\textit{angl. reward}), parenka sekantį veiksmą iš galimų veiksmų aibės ir taiko mokymosi algoritmą (algoritmas priklauso nuo agento implementacijos)
    \item atliekamas veiksmas -- perturbuojama programa arba požymių vektorius (priklauso
          nuo \glswhom{framework})
    \item gaunami aplinkos kitimo įverčiai -- nauja būsena ir atlygis, skaičiuojamas
          pagal detektoriaus klasifikacijos rezultatą
    \item seka kartojama tol, kol agentas nelaiko strategijos optimalia arba nustatytą
          kiekį kartų
\end{enumerate}
\citeplace{}

\begin{describeFramework}{DQEAF}{\cite{fangEvadingMalwareEngines2019}}
    \purpose{Parodyti, jog \acs{ml} kenkėjiškų programų detektoriai, ypač modeliai, treniruoti prižiūrimu mokymusi, yra pažeidžiami \glswhom{adversarial}}
    \surrogate{\acs{rl} karkasuose nenaudojami surogatiniai modeliai. Kaip \enquote{juodos dėžės} detektorius pasirinktas \acs{gbdt} modelis.}
    \mainModel{Agentas implementuotas kaip gilusis $Q$-tinklas (\acs{cnn} praplėtimas, kai tinklas naudojamas kaip \glswhom{qfunction} aproksimacija). Taip pat taikomas prioritetizuotas patirčių pakartojimo metodas (\textit{angl. prioritized experience replay}), kuomet agentas treniruojamas tik su aukštą atlygį gavusiais perėjimais ($S \times A$).}
    \features{Požymių vektorius taip pat apibrėžia visų būsenų erdvę $S$. Šiuo atveju $S = \mathbb{R}^{513}$.\vspace{5pt}}{
        \item Baitų lygio požymiai (\ref{sec:literature:features:byte}) -- baitų/entropijos histograma.
    }
    \perturbations{Perturbacijos apibrėžia visų galimų agento veiksmų erdvę $A$. Šiuo atveju $A = \set{0,1}^4$.\vspace{-10pt}}{
        \item Baitų lygio perturbacijos (\ref{sec:literature:perturbations:byte})
        \begin{itemize}
            \item \enquote{ARBE}
            \item \enquote{ARI}
            \item \enquote{ARS}
            \item \enquote{RS}
        \end{itemize}
    }
\end{describeFramework}

\begin{describeFramework}{Andersono}{\cite{andersonLearningEvadeStatic2018}}
\purpose{}
\surrogate{}
\mainModel{}
% \features{}{}
% \perturbations{}{}
\end{describeFramework}

\begin{describeFramework}{MalInfo}{\cite{zhongReinforcementLearningBased2022}}
\purpose{}
\surrogate{}
\mainModel{}
% \features{}{}
% \perturbations{}{}
\end{describeFramework}
\subsection{Genetinių algoritmų tipo modelių \glspl{framework}}\label{sec:literature:genetic}

Genetinai algoritmai (\acs{genetic}) yra viena seniausių mašininio mokymosi apraiškų; jų veikimas paremtas evoliucija \citeplace. Kenkėjiškų programų obfuskacijai \acs{ae} generavimas taikant \acs{genetic} yra tokia seka:
\begin{enumerate}
    \item sukuriama pradinė populiacija (perturbacijos metodai pradinei populiacijai priklauso nuo \glswhom{framework})
    \item atliekamas vertinimas naudojant detektorių\label{enum:genetic:eval}
    \item Jei vertinimo metu nustatoma, jog \acs{ae} yra pakankamai geros kokybės (pvz. detektorius klasifikuoja kaip nekenksmingą), seka baigiama
    \item atliekama selekcija -- dažniausiai pasirenkami geriausiai įvertinti populiacijos \acs{ae}, tačiau galimos ir kitos selekcijos strategijos
    \item atliekamas selekcijos atrinktų \acs{ae} kryžminimas (po 2) taip sukuriant naują \acs{ae}, turintį po dalį genų iš abiejų kryžmintų \acs{ae}
    \item tam tikrai daliai \ac{ae} atliekama dalies genų mutacija
    \item gaunama nauja populiacijos karta ir seka kartojama nuo \ref{enum:genetic:eval}-o žingsnio
\end{enumerate}
\cite{yusteOptimizationCodeCaves2022}
\subsection{Nevalidaus PE formato problema}\label{sec:literature:pe_invalid}
