\sectionnonum{Įvadas}

Pastaraisiais metais kenkėjiškas kodas ir programos kuriamos itin sparčiai (\sim450000 kenkėjiškų programų per dieną \href{https://www.av-test.org/en/statistics/malware/}{2024 m. AV-TEST}\footnote{https://www.av-test.org/en/statistics/malware} duomenimis). Kenkėjiško kodo aptikimo programos, kurios tradiciškai remiasi programų \gls{signature}, nespėja atnaujinti pėdsakų duomenų bazių pakankamai greitai. Dėl to \acfp{di}, tiksliau mašininio mokymosi (\acs{ml}), naudojimas kenkėjiškų programų ar kenkėjiško kodo aptikimo srityje tapo itin populiarus \cite{demetrioAdversarialEXEmplesSurvey2021}. Tačiau \ac{ml} modeliai, nors ir geba aptikti kenkėjiškas programas iš naujų, dar nematytų, duomenų, yra pažeidžiami \gls{adversarial} \cite{castroAIMEDEvolvingMalware2019,huGeneratingAdversarialMalware2017,rosenbergGenericBlackBoxEndEnd2018,zhongReinforcementLearningBased2022}. Šių atakų principas yra \ac{ml} modelio --~klasifikatoriaus~-- sprendimų priėmimo ribos \gls{decisionBoundary} radimas -- žinant šią ribą pakanka pakeisti kenkėjiškos programos veikimą taip, kad \ac{ml} modelis priimtų sprendimą klasifikuoti ją kaip nekenksmingą \cite{demetrioAdversarialEXEmplesSurvey2021}. Žinoma, rasti šią ribą nėra trivialus uždavinys. Mokslinėje literatūroje išskiriami 3 ribos paieškos atvejai \cite{fangEvadingMalwareEngines2019}:
\begin{enumerate}
    \item \textbf{Baltos dėžės} atvejis: kenkėjiško kodo kūrėjas turi visą informaciją apie \ac{ml} modelį, t.~y. modelio architektūrą, svorius, hiperparametrus.
    \item \textbf{Juodos dėžės su pasitikėjimo įverčiu} atvejis: kenkėjiško kodo kūrėjas gali tik testuoti modelį -- t.~y. pateikti programą ir gauti atsakymą. Atsakymo forma -- klasifikacija ir tikimybė, kad klasifikacija yra teisinga (pasitikėjimo įvertis).
    \item \textbf{Juodos dėžės} atvejis: kenkėjiško kodo kūrėjas gali tik testuoti modelį. Atsakymo forma yra tik klasifikacija.
\end{enumerate}
Akivaizdu, jog \enquote{juodos dėžės} atvejis yra sudėtingiausias, bet ir labiausiai atitinka realias sąlygas \cite{andersonLearningEvadeStatic2018}. Todėl šiame darbe nagrinėjami modeliai, gebantys generuoti varžymosi principais pagrįstų atakų obfuskuotus kenkėjiško kodo pavyzdžius (\acs{ae}) \enquote{juodos dėžės} atvejams.

%  TODO: black box / white box attacks and literature
%  TODO: actualize problem
\vspace{10pt}
\textbf{Tikslas} -- nustatyti labiausiai tinkantį modelį varžymosi principais pagrįstoms atakoms \enquote{juodos dėžės} atvejais.

\vspace{10pt}
\textbf{Uždaviniai}:
\begin{enumerate}
    \item Apžvelgti kenkėjiško kodo obfuskacijos metodus.
    \item Nustatyti kriterijus varžymosi principais grįstų atakų modeliams ir juos įvertinti.
    \item Atlikti eksperimentinį tyrimą su vienu iš įvertintų karkasų ir patikrinti vertinimo rezultatus.
\end{enumerate}