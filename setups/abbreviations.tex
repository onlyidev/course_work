% https://ctan.math.illinois.edu/macros/latex/contrib/acro/acro-manual.pdf

\acsetup{make-links=true,pages/display=none,list/template=supertabular,pages/seq/use=false,pages/fill={ }}

\DeclareAcronym{di}{
  short = DI,
  long = {dirbtinis intelektas},
  extra={\angl{artificial Intelligence}},
  short-plural-form = DI,
  long-plural-form ={dirbtinio intelekto}
  % highjack plural form for possesive, because plural will never be used
}

\DeclareAcronym{ml}{
  short=ML,
  long={mašininis mokymasis},
  extra={\angl{machine learning}}
}

\DeclareAcronym{gan}{
  short=GAN,
  long ={generatyviniai priešiški tinklai (\angl{generative adversarial networks})},
  long-plural-form={\angl{generative adversarial networks}}, % highjack plural for english
  extra={generatyviniai varžymosi principais pagrįsti tinklai}
}

\DeclareAcronym{ae}{
  short=AE,
  long={varžymosi principais pagrįstomis atakomis obfuskuoti kenkėjiško kodo pavyzdžiai},
  extra={\angl{adversarial examples}}
}

\DeclareAcronym{genetic}{
  short=GA,
  long={genetiniais algoritmais pagrįstas \ac{ml} modelis},
  extra={\angl{genetic algorithms}}
}

\DeclareAcronym{rl}{
  short=RL,
  long={skatinamasis mokymasis},
  extra={\angl{reinforcement learning}}
}

\DeclareAcronym{dll}{
  short=DLL,
  long={dinamiškai susieta biblioteka},
  extra={\angl{dyanmic-link library}}
}

\DeclareAcronym{pe}{
  short=PE,
  long={\angl{portable executable}}
}

\DeclareAcronym{nlp}{
  short=NLP,
  long={skaitmeninis natūraliosios kalbos apdorojimas},
  extra={\angl{natural language processing}}
}

\DeclareAcronym{api}{
  short=API,
  long={\angl{application programming interface}}
}

\DeclareAcronym{svm}{
  short=SVM,
  long={\angl{support vector machine}}
}

\DeclareAcronym{gbdt}{
  short=GBDT,
  long={\angl{gradient boosted decision trees}}
}

\DeclareAcronym{knn}{
  short=KNN,
  long={\angl{$K$-Nearest Neighbours}}
}

\DeclareAcronym{cnn}{
  short=CNN,
  long={\angl{convolutional neural network}}
}