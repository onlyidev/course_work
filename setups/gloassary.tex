% \glsaddkey
%  {plural-what}% key
%  {}% default value
%  {\glsentryplwhat}% no link cs
%  {\Glsentryplwhat}% no link ucfirst cs
%  {\glsplwhat}% link cs
%  {\Glsplwhat}% link ucfirst cs
%  {\GLSplwhat}% link all caps cs

\newglossaryentry{signature}{
    name = Signature,
    description = {Kodo/programos pėdsakas},
    text=pėdsakas,
    first = {pėdsakais (\textit{angl. signature})},
}

\newglossaryentry{adversarial}{
    name = Adversarial attacks,
    description = {Varžymosi principais pagrįstos atakos},
    text=varžymosi principais pagrįstos atakos,
    first = {varžymosi principais pagrįstoms atakoms (\textit{angl. adversarial attacks})},
}

\newglossaryentry{decisionBoundary}{
    name = Decision Boundary,
    description = {Sprendimų priėmimo riba},
    text=sprendimų priėmimo riba,
    first = {(\textit{angl. decision boundary})},
}

\newglossaryentry{zeroSumGame}{
    name={Zero-sum Game},
    description={Dviejų žaidėjų žaidimas, kuriame galimas vienas laimėtojas. Laimėtojo laimėta suma yra lygi pralaimėtojo pralaimėtai sumai},
    text={nulinės sumos žaidimas},
    first={nulinės sumos žaidimą (\textit{angl. zero-sum game})}
}

\newglossaryentry{surrogateModel}{
    name={Surogatinis Modelis},
    description={\ac{ml} modelis, aproksimuojantis kitą \ac{ml} modelį, kurio parametrai (svoriai) nėra žinomi},
    text={surogatinis modelis},
    first={surogatinio modelio}
}

\newglossaryentry{policy}{
    name=Strategija,
    description={\textit{angl. Policy}. \acs{rl} modelio atliekama veiksmų seka},
    text={\textit{angl. policy}}
}