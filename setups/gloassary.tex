% \glsaddkey
%  {plural-what}% key
%  {}% default value
%  {\glsentryplwhat}% no link cs
%  {\Glsentryplwhat}% no link ucfirst cs
%  {\glsplwhat}% link cs
%  {\Glsplwhat}% link ucfirst cs
%  {\GLSplwhat}% link all caps cs

\glsaddkey{what}% key
{}% default value
{\glsentrywhat}% no link cs
{\Glsentrywhat}% no link ucfirst cs
{\glswhat}% link cs
{\Glswhat}% link ucfirst cs
{\GLSwhat}% link all caps cs

\glsaddkey{whom}% key
{}% default value
{\glsentrywhom}% no link cs
{\Glsentrywhom}% no link ucfirst cs
{\glswhom}% link cs
{\Glswhom}% link ucfirst cs
{\GLSwhom}% link all caps cs
\glsaddkey{whose}% key
{}% default value
{\glsentrywhose}% no link cs
{\Glsentrywhose}% no link ucfirst cs
{\glswhose}% link cs
{\Glswhose}% link ucfirst cs
{\GLSwhose}% link all caps cs

\glsaddkey{plural-whom}% key
{}% default value
{\glsentryplwhom}% no link cs
{\Glsentryplwhom}% no link ucfirst cs
{\glsplwhom}% link cs
{\Glsplwhom}% link ucfirst cs
{\GLSplwhom}% link all caps cs

\glsaddkey{plural-what}% key
{}% default value
{\glsentryplwhat}% no link cs
{\Glsentryplwhat}% no link ucfirst cs
{\glsplwhat}% link cs
{\Glsplwhat}% link ucfirst cs
{\GLSplwhat}% link all caps cs

\newcommand{\glsshort}[1]{\glslink{#1}{\glsentryshort{#1}}}
\newcommand{\Glsshort}[1]{\glslink{#1}{\Glsentryshort{#1}}}

\newglossaryentry{signature}{
    name={Pėdsakas (\angl{Signature})},
    description={Programos struktūros ir požymių santrauka, beveik unikaliai identifikuojanti programą (pvz. \gls{hashfunction})},
    text=pėdsakas,
    first = {pėdsakais \angl{signature}},
}

\newglossaryentry{adversarial}{
    name={Varžymosi principais pagrįstos atakos (\angl{Adversarial Attacks})},
    description={Tai atakos, pritaikytos \enquote{apgauti} \acs{ml} klasifikatorius},
    text=varžymosi principais pagrįstos atakos,
    what=varžymosi principais pagrįstas atakas,
    whom={varžymosi principais pagrįstoms atakoms},
    whose={varžymosi principais pagrįstų atakų},
    first = {varžymosi principais pagrįstoms atakoms \angl{adversarial attacks}},
}

\newglossaryentry{decisionBoundary}{
    name={Sprendimų priėmimo riba (\angl{Decision Boundary})},
    description={Paprasčiausiems \acs{ml} modeliams tai yra kreivė plokštumoje. Sudėtingesniems -- daugiadimensiniams modeliams -- daugdara (\angl{manifold})},
    text=sprendimų priėmimo riba,
    first = {\angl{decision boundary}},
}

\newglossaryentry{zeroSumGame}{
    name={Nulinės sumos žaidimas (\angl{Zero-Sum Game})},
    description={Dviejų žaidėjų žaidimas, kuriame galimas vienas laimėtojas. Laimėtojo laimėta suma yra lygi pralaimėtojo pralaimėtai sumai},
    text={nulinės sumos žaidimas},
    first={nulinės sumos žaidimą \angl{zero-sum game}}
}

\newglossaryentry{surrogateModel}{
    name={Surogatinis Modelis (\angl{Surrogate Model})},
    description={\ac{ml} modelis, aproksimuojantis kitą \ac{ml} modelį, kurio parametrai (svoriai) nėra žinomi},
    text={surogatinis modelis},
    first={surogatinio modelio},
    whom={surogatinio modelio},
    what={surogatinį modelį}
}

\newglossaryentry{policy}{
    name={Strategija (\angl{Policy})},
    description={Tai funkcija $\pi : S \times A \rightarrow \set{0,1}$, čia $S$ -- galimų būsenų erdvė (\angl{State Space}), $A$ -- galimų veiksmų erdvė (\angl{Action Space}). Šią funkciją \acs{rl} modelis \enquote{išmoksta} mokymosi metu},
    text={\angl{policy}}
}

\newglossaryentry{hashfunction}{
    name={Maišymo Funkcija (\angl{Hash Function})},
    description={Tai funkcija $f: \set{0,1}^* \rightarrow \set{0,1}^m$. Naudojama, kai iš begalinės įvesčių erdvės norima gauti fiksuoto dydžio ($m$) išvestį},
    text={maišymo funkcija}
}

\newglossaryentry{framework}{
    name={Karkasas (\angl{Framework})},
    description={Nurodo specifines technologijas, naudojamus požymius ir perturbacijas, siekiamus tikslus \acs{ae} generacijai. Skirtas apibrėžti procesą ir įrankius, kuriuos naudojant būtų galima generuoti nurodytų tikslų siekiančius \acs{ae}},
    text={varžymosi principais pagrįstų atakų karkasas},
    short=karkasas,
    what=karkasą,
    plural=karkasai,
    plural-whom=karkasų,
    plural-what=karkasus,
    whom=karkaso
}

\newglossaryentry{qfunction}{
    name={Q-Funkcija (\angl{Q-Function})},
    description={$Q: S \times A \rightarrow \mathbb{R}$, čia $S$ -- galimų būsenų erdvė (\angl{State Space}), $A$ -- galimų veiksmų erdvė (\angl{Action Space})},
    text=$Q$-funkcija,
    whom=$Q$-funkcijos
}