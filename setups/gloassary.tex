% \glsaddkey
%  {plural-what}% key
%  {}% default value
%  {\glsentryplwhat}% no link cs
%  {\Glsentryplwhat}% no link ucfirst cs
%  {\glsplwhat}% link cs
%  {\Glsplwhat}% link ucfirst cs
%  {\GLSplwhat}% link all caps cs

\glsaddkey{what}% key
{}% default value
{\glsentrywhat}% no link cs
{\Glsentrywhat}% no link ucfirst cs
{\glswhat}% link cs
{\Glswhat}% link ucfirst cs
{\GLSwhat}% link all caps cs

\glsaddkey{whom}% key
{}% default value
{\glsentrywhom}% no link cs
{\Glsentrywhom}% no link ucfirst cs
{\glswhom}% link cs
{\Glswhom}% link ucfirst cs
{\GLSwhom}% link all caps cs

\glsaddkey{plural-whom}% key
{}% default value
{\glsentryplwhom}% no link cs
{\Glsentryplwhom}% no link ucfirst cs
{\glsplwhom}% link cs
{\Glsplwhom}% link ucfirst cs
{\GLSplwhom}% link all caps cs

\newglossaryentry{signature}{
    name = Signature,
    description = {Kodo/programos pėdsakas},
    text=pėdsakas,
    first = {pėdsakais (\textit{angl. signature})},
}

\newglossaryentry{adversarial}{
    name = Adversarial attacks,
    description = {Varžymosi principais pagrįstos atakos},
    text=varžymosi principais pagrįstos atakos,
    what=varžymosi principais pagrįstas atakas,
    whom={varžymosi principais pagrįstoms atakoms},
    first = {varžymosi principais pagrįstoms atakoms (\textit{angl. adversarial attacks})},
}

\newglossaryentry{decisionBoundary}{
    name = Decision Boundary,
    description = {Sprendimų priėmimo riba. Paprasčiausiems \acs{ml} modeliams tai yra kreivė plokštumoje. Sudėtingesniems -- daugiadimensiniams modeliams -- daugdara (\textit{angl. manifold})},
    text=sprendimų priėmimo riba,
    first = {(\textit{angl. decision boundary})},
}

\newglossaryentry{zeroSumGame}{
    name={Zero-sum Game},
    description={Dviejų žaidėjų žaidimas, kuriame galimas vienas laimėtojas. Laimėtojo laimėta suma yra lygi pralaimėtojo pralaimėtai sumai},
    text={nulinės sumos žaidimas},
    first={nulinės sumos žaidimą (\textit{angl. zero-sum game})}
}

\newglossaryentry{surrogateModel}{
    name={Surogatinis Modelis},
    description={\ac{ml} modelis, aproksimuojantis kitą \ac{ml} modelį, kurio parametrai (svoriai) nėra žinomi},
    text={surogatinis modelis},
    first={surogatinio modelio},
    whom={surogatinio modelio},
}

\newglossaryentry{policy}{
    name=Strategija,
    description={\textit{angl. Policy}. Tai funkcija $\pi : S \times A \rightarrow \set{0,1}$, kur $A,S$ yra galimų veiksmų erdvė (\textit{angl. Action Space}) ir galimų būsenų erdvė (\textit{angl. State Space}) atitinkamai. Šią funkciją \acs{rl} modelis \enquote{išmoksta} mokymosi metu},
    text={\textit{angl. policy}}
}

\newglossaryentry{hashfunction}{
    name={Maišymo Funkcija},
    description={\textit{angl. Hash Function}. Tai funkcija $f: \set{0,1}^* \rightarrow \set{0,1}^m$. Naudojama, kai iš begalinės įvesčių erdvės norima gauti fiksuoto dydžio ($m$) išvestį},
    text={maišymo funkcija}
}

\newglossaryentry{framework}{
    name=Karkasas,
    description={Nurodo specifines technologijas, naudojamus požymius ir perturbacijas, siekiamus tikslus \acs{ae} generacijai. Skirtas apibrėžti procesą ir įrankius, kuriuos naudojant būtų galima generuoti nurodytų tikslų siekiančius \acs{ae}},
    text={varžymosi principais pagrįstų atakų karkasas},
    short=karkasas,
    what=karkasą,
    plural=karkasai,
    plural-whom=karkasų,
    whom=karkaso
}

\newglossaryentry{qfunction}{
    name=Q-Funkcija,
    description={$Q: S \times A \rightarrow \mathbb{R}$, kur $A,S$ yra galimų veiksmų erdvė (\textit{angl. Action Space}) ir galimų būsenų erdvė (\textit{angl. State Space}) atitinkamai},
    text=$Q$-funkcija,
    whom=$Q$-funkcijos
}