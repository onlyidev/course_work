\newenvironment{describeModel}[2]{
    \newcommand{\purpose}[1]{\gdef\Purpose{##1}}
    \newcommand{\surrogate}[1]{\gdef\Surrogate{##1}}
    \newcommand{\mainModel}[1]{\gdef\MainModel{##1}}
    \newcommand{\features}[1]{\gdef\Features{##1}}
    \newcommand{\perturbations}[1]{\gdef\Perturbations{##1}}

    \def\Purpose{}
    \def\Surrogate{}
    \def\MainModel{}
    \def\Features{\item}
    \def\Perturbations{\item}
    \def\Name{#1}
    \def\Citation{#2}
}{
    \subsubsection{\enquote{\Name}}\label{sec:literature:framework:\Name}
    % \begin{itemize}
    %     \item \textbf{Tikslas}: \Purpose{}
    %     \item \textbf{\Glswhom{surrogateModel} tipas ir architektūra}: \Surrogate{}
    %     \item \textbf{\acs{ml} modelio tipas ir architektūra}: \MainModel{}
    %     \item \textbf{Naudojami požymiai}: \Features{}
    %     \item \textbf{Naudojamos perturbacijos}: \Perturbations{}
    % \end{itemize}
    \begin{longtable}{p{0.15\textwidth}|p{0.85\textwidth}}
        \textbf{Tikslas}              & \Purpose{}                                     \\ \toprule
        \textbf{\Gls{surrogateModel}} & \Surrogate{}                                   \\ \midrule
        \textbf{\acs{ml} modelis}     & \MainModel{}                                   \\ \midrule
        \textbf{Požymiai}             & \begin{itemize} \Features{} \end{itemize}      \\ \midrule
        \textbf{Perturbacijos}        & \begin{itemize} \Perturbations{} \end{itemize} \\ \bottomrule
    \end{longtable}
    \Citation{}
}