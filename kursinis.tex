\documentclass[
    english, % Klasei padavus parametrą 'english', darbas bus anglų kalba.
    % signatureplaces % prideda parašų vietas tituliniame puslapyje
]{VUMIFPSkursinis}
\usepackage{float}
\usepackage{wrapfig2}
\usepackage{hyperref}
\usepackage{algorithmicx}
\usepackage{algorithm}
\usepackage{algpseudocode}
\usepackage{amsfonts}
\usepackage{amsmath}
\usepackage{bm}
\usepackage{caption}
\usepackage{color}
\usepackage{graphicx}
\usepackage{listings}
\usepackage{subcaption}
\usepackage{biblatex}

\university{Vilniaus universitetas}
\faculty{Matematikos ir informatikos fakultetas}
\department{Programų sistemų studijų programa}
\papertype{Research Paper}
\title{GAN architektūrų, tinkamų kenkėjiško kodo obfuskacijai, analizė}
\titleineng{Analysis of GAN architectures suitable for Ethical Malware Obfuscation}
\author{Liudas Kasperavičius}
\status{4 kurso 3 grupės studentas}
\supervisor{prof. dr. Olga Kurasova}
\reviewer{TBD}
\date{Vilnius – \the\year}

\bibliography{bibliography}

\begin{document}
\maketitle

\tableofcontents

\sectionnonum{Notes}

\begin{itemize}
    \item \url{notes/1.md} - Notes about \cite{nguyenGenerativeAdversarialNetworks2023}
    \item \url{notes/2.md} - Notes about \cite{zhongMalFoxCamouflagedAdversarial2024}
    \item \url{notes/3.md} - Notes about \cite{zhongReinforcementLearningBased2022}
    \item \url{notes/4.md} - Notes about \cite{kawaiImprovedMalGANAvoiding2019}
    \item \url{notes/5.md} - Notes about \cite{huGeneratingAdversarialMalware2017}
\end{itemize}

\sectionnonum{Įvadas}

Pastaraisiais metais kenkėjiškas kodas ir programos kuriamos itin sparčiai (\sim450000 kenkėjiškų programų per dieną \href{https://www.av-test.org/en/statistics/malware/}{AV-TEST} duomenimis). Kenkėjiško kodo aptikimo programos, kurios tradiciškai remiasi programų \gls{signature}, nespėja atnaujinti pėdsakų duomenų bazių pakankamai greitai. Dėl to \acfp{di}, tiksliau Mašininio Mokymosi (\acs{ml}), naudojimas kenkėjiškų programų ar kenkėjiško kodo aptikimo srityje tapo itin populiarus \cite{demetrioAdversarialEXEmplesSurvey2021}. Tačiau \ac{ml} modeliai, nors ir geba aptikti kenkėjiškas programas iš naujų, dar nematytų, duomenų, yra pažeidžiami \gls{adversarial} \cite{castroAIMEDEvolvingMalware2019,huGeneratingAdversarialMalware2017,rosenbergGenericBlackBoxEndEnd2018,zhongReinforcementLearningBased2022}. Šių atakų principas yra \ac{ml} modelio sprendimų priėmimo ribos \gls{decisionBoundary} radimas -- žinant šią ribą pakanka pakeisti kenkėjiškos programos veikimą taip, kad \ac{ml} modelis priimtų sprendimą klasifikuoti ją kaip nekenksmingą \cite{demetrioAdversarialEXEmplesSurvey2021}. Žinoma, rasti šią ribą nėra trivialus uždavinys. Mokslinėje literatūroje išskiriami 3 ribos paieškos atvejai \cite{fangEvadingMalwareEngines2019}:
\begin{enumerate}
    \item \textbf{Baltos dėžės} atvejis: kenkėjiško kodo kūrėjas turi visą informaciją apie \ac{ml} modelį, t.~y. modelio architektūrą, svorius, hiperparametrus.
    \item \textbf{Juodos dėžės su pasitikėjimo įverčiu} atvejis: kenkėjiško kodo kūrėjas gali tik testuoti modelį -- t.~y. pateikti programą ir gauti atsakymą. Atsakymo forma -- klasifikacija ir tikimybė, kad klasifikacija yra teisinga (pasitikėjimo įvertis).
    \item \textbf{Juodos dėžės} atvejis: kenkėjiško kodo kūrėjas gali tik testuoti modelį. Atsakymo forma yra tik klasifikacija.
\end{enumerate}
Akivaizdu, jog \enquote{juodos dėžės} atvejis yra sudėtingiausias, bet ir labiausiai atitinka realias sąlygas \citeplace. Todėl šiame darbe nagrinėjami modeliai, gebantys generuoti varžymosi principais pagrįstų atakų obfuskuotus kenkėjiško kodo pavyzdžius (\acs{ae}) \enquote{juodos dėžės} atvejams.

%  TODO: black box / white box attacks and literature
%  TODO: actualize problem
\vspace{10pt}
\textbf{Tikslas} -- nustatyti labiausiai tinkantį modelį varžymosi principais pagrįstoms atakoms \enquote{juodos dėžės} atvejais.

\vspace{10pt}
\textbf{Uždaviniai}:
\begin{enumerate}
    \item Apžvelgti kenkėjiško kodo obfuskacijos metodus
    \item Nustatyti kriterijus varžymosi principais grįstų atakų modeliams ir juos įvertinti
    \item Atlikti eksperimentinį tyrimą naudojant modelį, gavusį aukštą įvertinimą pagal kriterijus
\end{enumerate}
\sectionnonum{Rezultatai ir išvados}

\subsection*{Rezultatai}
\begin{enumerate}
    \item Išanalizuoti 8 trims kategorijoms (\acs{gan}, \acs{rl}, \acs{genetic}) priklausantys \glswhose{adversarial} \glspl{framework}. 
    \item Jiems apibrėžti vertinimo kriterijai, išskiriantys geriausiai realioms sąlygoms pritaikytus \glsplwhat{framework}.
    \item Atliktas tyrimas, kurio metu \ref{enum:criteria:commercial} kriterijaus neįgyvendinantis \gls{framework} \refFramework{MalGAN} naudojamas \acs{ae} generavimui prieš komercinius detektorius ir renkami \ref{enum:criteria:effective} kriterijų atitinkantys duomenys (t.~y. nustatomas \glswhom{framework} \ref{enum:criteria:effective} kriterijus tokiomis sąlygomis, kai tenkinamas \ref{enum:criteria:commercial}).
\end{enumerate}

\subsection*{Išvados}
\begin{enumerate}
    \item Tyrimo metu atlikti 2 eksperimentai parodo, jog net ir padidinus \textit{MalGAN} \acs{ml} modelio galimybes (suteikus daugiau kompiuterijos resursų), atakų efektyvumas prieš komercinius detektorius nepadidėja ($\bar{\mu}_{E \text{-} 2} < \bar{\mu}_{E \text{-} 1}$), tačiau yra stabilesnis ($\sigma_{E \text{-} 2} < \sigma_{E \text{-} 1}$).
    \item Abiem atvejais tyrimo metu gautas \ref{enum:criteria:effective} įvertis ($\bar{\mu}$) yra žymiai mažesnis už kriterijų vertinimo metu naudojamą $81,00 \; \%$ \textit{MalGAN} \glswhom{framework} įvertinimą (kai \ref{enum:criteria:commercial} netenkinamas). Tai paaiškina \ref{tab:criteria}-oje lentelėje matomas \ref{enum:criteria:effective} kriterijaus anomalijas ir parodo, jog kriterijai ir \glsplwhom{framework} vertinimas yra adekvatūs.
    \item Laikant, jog parinkti kriterijai ir jų vertinimas yra patikimi, geriausiai iš analizuotų \glsplwhom{framework} \enquote{juodos dėžės} atvejams pritaikytas \gls{framework} yra \refFramework{MalFox}, tenkinantis visus 3 kokybinius kriterijus ir siekiantis $56,00 \; \%$ kiekybinį įvertinimą (atakų efektyvumą).
\end{enumerate}

\sectionnonum{Conclusions}
\begin{enumerate}[labelindent=0pt]
    \item The conclusions section compares the methods for solving the examined problems, offers recommendations, and highlights innovations.
    \item Conclusions are presented in a numbered (possibly hierarchical) list format.
    \item The conclusions of the work must correspond to the aim of the work.
\end{enumerate}

\printbibliography[heading=bibintoc]

% \sectionnonum{Sąvokų apibrėžimai}
% \sectionnonum{Santrumpos}

% \appendix{Appendix}

\end{document}
